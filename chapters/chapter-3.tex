\chapter{Results} \label{chapter3}

This is more text, see~\cite{utk:idr2015gpu}.

\begin{table}[!htb]
    \Centering
    \caption[Table with multiple rows]{A multirow table example.}
    \begin{tabular}{|L{3cm}|C{1cm}|C{1cm}|}
        \hline
        \textbf{col1} & \textbf{col2} & \textbf{col3} \\
        \hline
        \multirow{3}{3cm}{Multiple rows}
            & cell2 & cell3 \\
            & cell5 & cell6 \\
            & cell8 & cell9 \\
        \hline
    \end{tabular}
    \label{tab:multirow2}
\end{table}

Discussing some analysis results from~\autoref{tab:multirow2}.

\section{Plots} \label{plots}

\begin{figure}[!htb]
    \Centering
    \begin{subfigure}[t]{0.45\textwidth}
        \Centering
        \includegraphics[width=1in]{fig02a-circle}
        \caption{\ Circle}
        \label{fig:shapes-circle2}
    \end{subfigure}
    \begin{subfigure}[t]{0.45\textwidth}
        \Centering
        \includegraphics[width=1in]{fig02b-rectangle}
        \caption{\ Rectangle}
        \label{fig:shapes-rect2}
    \end{subfigure}
    \caption[Geometric shapes]{Geometric shapes, each presented as a subfigure.
        (a) is a circle and
        (b) is a rectangle}
    \label{fig:shapes2}
\end{figure}

For multipart figures (e.g.,~\autoref{fig:shapes2}),
you need to use the package ``subcaption''.
